\label{sec:data}
% Main characteristics of the data set: source, type of data
% Description of variables used for the	analysis and correspondence with the (ideal) magnitudes in the empirical specification
% Descriptive statistics of the	main variables in the analysis
We have scraped most of our data from various web sources in respect of the terms of use. The descriptive statistics for the main variables are shown in table \ref{tab:descriptive} and described in further detail in the remainder of this section.
\medskip\\
For estimation purposes we log-transform electricity consumption, the number of electricity meters, and the electricity spot price. This furthermore allows us to interpret parameter estimates as elasticities/semi-elasticities. Before taking the natural logarithm the variables are censored with $1$ as the lower bound whereby we loose some information as the spot price is negative for a few instances due to surplus wind power production.
\begin{table}[H]
  \centering
  \caption{Descriptive statistics}
  \label{tab:descriptive}
  \footnotesize
    \begin{tabular}{l*{1}{ccccc}}
\hline\hline
                    &\multicolumn{5}{c}{}                                            \\
                    &        mean&          sd&         min&         p50&         max\\
\midrule
Wholesale electricity use&     34694.1&    84721.53&      63.157&      5928.1&    757557.1\\
Household electricity use&    29160.88&    75328.71&      141.67&    5833.974&    906396.4\\
Number of wholesale meters&    930.5284&    2482.449&           7&       140.5&       17674\\
Number of retail meters&    57736.69&      150765&         858&       14371&     1006061\\
- of which flex-settled&    4150.063&    37655.51&           0&           0&      596267\\
- of which residual &    53586.63&    136813.6&         855&       14357&      998864\\
Electricity spot price&    253.0939&     108.095&     -398.61&         235&      1898.9\\
Wind power prognosis same region&    1069.817&    911.7859&           0&         787&        3973\\
Wind power prognosis other region&    482.2171&    568.4732&           0&         313&        3973\\
Price region DK1    &    .8269231&    .3783139&           0&           1&           1\\
Time-of-use tariff  &    .0001431&    .0081221&           0&           0&    .5926748\\
Temperature         &    9.101392&    6.918248&       -11.9&         8.7&        31.4\\
Daytime             &    .5135293&    .4857424&           0&    .6666667&           1\\
Time trend          &    547.4559&    316.4119&           0&         547&        1095\\
Holiday (not in a weekend)&    .0437643&    .2045702&           0&           0&           1\\
\midrule
Observations        &     1367600&            &            &            &            \\
\bottomrule

\end{table}

\subsection{Grid-level consumption}
\label{subsec:d_consumption}
The Danish Transmission System Operator (TSO), Energinet provides public access to hourly aggregated consumption data\footnote{Scraped from \href{https://www.energidataservice.dk/en/dataset/consumptionpergridarea/}{energidataservice.dk/en/dataset/consumptionpergridarea} using their transparent API via SQL statements.} since January 2016 for each grid company grouped by hourly-settled consumption, flex settled consumption, and residual consumption. This allows us to distinguish between wholesale and retail consumption. Hourly-settled consumption consists of all firms with an annual electricity consumption of at least 100,000 kWh.
\par
Flex-settled consumption was introduced in January 2018 such that households and small firms can opt to have their electricity consumption settled more flexibly for example according to real time electricity prices. Though installation of smart meters to enable flex-settling is only being introduced gradually, this allows a portion of residential consumers and small firms to better respond to price changes at an hourly rather than a yearly or quarterly basis.
\par
The residual consumption is the remaining retail electricity consumption for which flex-settling is not used and thus includes all households and small firms till December 1 2017 and throughout the majority of 2018 as well.
\begin{figure}[H]
  \centering
  \caption{Mean electricity consumption by hour and type}
    \label{fig:cons_hours}
  \includegraphics[width=1 \textwidth]{03_figures/cons_hours}
\end{figure}

\begin{figure}[H]
  \centering
  \caption{Time series for mean electricity consumption (business days)}
  \label{fig:cons_time_series}
  \includegraphics[width=1 \textwidth]{03_figures/cons_time series, business days}
\end{figure}
\noindent
As shown in figure \ref{fig:cons_hours} and \ref{fig:cons_time_series} wholesale and retail consumption follows clear patterns not only within the day but also between days and across the year. That is, wholesale consumption is at its lowest on weekends, bank holidays and during the summer holiday while retail consumption peaks in the hours 5-7 PM (from here on written as hours 17-19) and half of the year during the winter.
\medskip\\
For each grid we include the number of metering points\footnote{Received from Energinet after request.} by each of the three consumer categories. \todo{Hvorfor gør vi det? } This data is monthly by the \nth{1} of the month. For studies on state-level data it is likewise common to control for size \citep{burke2017price}.
\medskip\\
The landscape of grid companies has changed drastically. From consisting of 74 grid companies by early 2016, only 56 grid companies remained by the end of 2018 (see figure \ref{fig:elnetgraenser} in appendix \ref{app:grids}). We remove the two grids with less than 10 metering points and the six grids with no wholesale consumption, which leaves us with 48 grids of which no less than 39 are located in Western Denmark. For a merged grid company we apply the sum of each of the grids included in The future merge to all prior month as described further in appendix \ref{app:grids}.\todo{@Thor - Er det korrekt forstået, som jeg har omskrevet}. 


\subsection{Spot market prices and wind power prognosis}
\label{subsec:d_spot}
We include the hour-by-hour spot market price on the day-ahead-market for the price region DK1 (Western Denmark) or DK2 (Eastern Denmark) depending on where the grid company is located (see section \ref{sec:theory}). An important factor for the spot price on the day-ahead-market is the hour-by-hour wind power prognosis for the following day.\footnote{'Elspot prices' and 'Wind power prognosis' by price region and year is updated daily after 2PM by Nord Pool and downloadable at \href{https://www.nordpoolgroup.com/historical-market-data/}{nordpoolgroup.com/historical-market-data}} While being less volatile than wind power production, price is nonetheless highly volatile from day to day while having increased in 2018 as illustrated by the time series in figure \ref{fig:wp_dk1_time_series} and \ref{fig:wp_dk2_time_series} (appendix \ref{app:data}).
\medskip\\
The wind power prognosis first and foremost takes into account weather forecasts in relation to the positions and capacity of windmills but also takes into account the expected demand as some wind mills can possibly be turned off if the expected price is too low. However, except for a slight peak in the afternoon and evening that is more likely due to sea and land breezes, wind power does not seem to care much for consumption patterns during the day (figure \ref{fig:trio_DK2_hours} for DK2 and \ref{fig:wp_dk1_hours} for DK1) and especially not across weekdays (figure \ref{fig:wp_price_weekday}).
\par
On the contrary, the daily pattern of the spot price on average follows the pattern of demand by and large. The biggest gap between price and total consumption seen in figure \ref{fig:trio_DK2_hours} occurs during the afternoon where the low price relative to demand could possibly be explained by the higher wind power production.
\begin{figure}[H]
  \centering
  \caption{Total consumption, wind power and spot price by hour (business days)}
  \label{fig:trio_DK2_hours}
    \includegraphics[width=1 \textwidth]{03_figures/trio_DK2_hours, business days}
\end{figure}


\subsection{Time-of-use tariff}
\label{subsec:d_tout}
Since December 2017 grid companies have been allowed to introduce time-of-use (TOU) tariffs for retail consumption in order to send signals to encourage shifts of flexible tasks away from the peak hours around dinnertime. Two of the bigger grid companies have already introduced TOU tariffs for the peak-hours 17-19 for the months October-March in which electricity consumption is also higher due to the lack of daylight. While Konstant initially only runs an experiment for a smaller group of flex-settled consumers, Radius is introducing a full-scale TOU tariff scheme while exchanging the old prepayment meters with smart meters for the 600,000 retail customers in the Copenhagen metropolitan area.\footnote{See \href{https://ing.dk/artikel/nu-loebes-fleksible-elforbrug-omsider-gang-209251}{ing.dk/artikel/nu-loebes-fleksible-elforbrug-omsider-gang-209251} (Danish).}
\medskip\\
The variable for the TOU tariff represents the share of retail customers in Radius exposed to the tariff. As seen in figure \ref{fig:radius_w39_w40} and table \ref{tab:descriptive} below the share increases throughout the period and ends near 60 percent in December 2018. The concept of aggregate data makes it difficult to demarcate changes in behavior from changes in composition. In figure \ref{fig:radius_w39_w40} we try to investigate the discontinuity around October 1 2018 as week 39 is in September and week 40 is in October. From a graphical inspection no clear response to the TOU tariff stands out, except that flex-settled consumers have higher consumption during the day and residual consumers during the night which might be due to sociodemographic differences between the areas with smart meters and those where it has yet to be implemented.
\begin{figure}[H]
  \centering
  \caption{Flex-settled and residual consumption by week (Radius, 2018)}
  \label{fig:radius_w39_w40}
      \includegraphics[width=1 \textwidth]{03_figures/radius_w39_w40}
\end{figure}

\subsection{Weather data}
\label{subsec:d_weather}
The outside temperature is relevant to the extent that electrical heaters or air conditioning is used \citep{lijesen2007real, vesterberg2014residential}. As the electricity consumption ceteris paribus is expected to be higher for both low-end and high-end temperatures, we let the effect of temperature enter as a \nth{2} order polynomial in the estimation of electricity consumption.\footnote{Scraped via iterative lookups in the records of the Danish Meteorological Institute at \href{https://www.dmi.dk/vejrarkiv/}{dmi.dk/vejrarkiv/}}
\medskip\\
Lighting is used more in the absence of daylight. Therefore, an indicator for daytime is included such that $daytime=1$ for hours between sunrise and sunset and e.g. $daytime=0.25$ for $hour=7$ if sunrise was a quarter past 7.\footnote{Sunrise and sunset are scraped for each date in the sample via iterative lookups at \href{https://soltider.dk/}{soltider.dk}}
\medskip\\
Taking advantage of the population density in Denmark, temperature and daytime are only scraped for the two most populous municipalities (Aarhus and Copenhagen) and then applied to all grid companies within their respective price regions.\footnote{Temperature is for the municipalities of Aarhus and Copenhagen respectively while sunrise and sunset are for the City Hall Square in each of the two cites.}

\subsection{Time controls}
\label{subsec:d_time}
Year dummies as well as a time trend indicating the number of days since January 1 2016 are included to account for economic growth (overall increases in electricity consumption), technological progress (decreases in electricity consumption per appliance) \citep{lijesen2007real}, or other compositional changes that can affect electricity consumption other than the number of meters.
\medskip\\
Danish bank holidays and a few other common holidays with lower wholesale electricity consumption\footnote{January 2 (the day after New Year's Day), May 1 (International Workers' Day), Friday after Ascension Day, June 5 (Constitution Day), last Friday before Christmas, and the days between Christmas and New Year's. All holidays according to \href{https://kalendersiden.dk/}{kalendersiden.dk}} are taken into account in order to do sample split regressions for business days and non-business days, the latter including the aforementioned holidays and weekends.
