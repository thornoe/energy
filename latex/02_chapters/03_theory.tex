\label{sec:theory}
% Theoretical arguments in the literature closely related to your study
\subsection{Price formation in the spot market for electricity}
\label{subsec:t_price}


\subsection{Simultaneity of demand and supply}
\label{subsec:t_simultaneity}
Our data set contains all firms that have their hourly consumption charged by the corresponding spot market price, however, for big companies options exist to negotiate a less volatile settling or to buy financial products in order to increase price security. Our data does not cover firms that buy their electricity directly on the spot market or bilaterally by the electricity supplier on the so-called over-the-counter (OTC) market. It is a clear advantage that we know all of the demand in our firm data is directly subject to the spot price \citep{lijesen2007real}.


\subsection{Theory of demand-side response to electricity prices}
\label{subsec:t_demand}
Indirect demand - through the use of other appliances that demand for electricity is shaped. Not much information about consumption available => people adjust their consumption according to behavioural rules. t

Two obvious limitations to the flexibility of demand is that demand responses require knowledge of prices and that one can assume taking action in order to reduce demand at a certain hour carries a fixed cost of e.g. \$5 which makes is optimal for small consumers to ignore spot market except at extreme prices or when time-of-use (TOU) tariffs are introduces \citep{wolak2011residential}. 

Another relevant point to make is that residential electricity consumers historically have not been treated as genuine 'demanders' in the sense that they are insulated from the spot price and instead face a fixed price that does not reflect the cost of generation of electricity at a given point in time. This is exacerbated even more by the multitude of tariffs that residential consumers face.  
In Denmark these are particularly high and hence we may except an even smaller price sensitivity from the residential consumers in our estimation.   

This all adds up to a weak elasticity. Another aspect has to do with how electricity is regarded as a good. \citep{kirschen2003demand} points to the fact that electricity is regarded as a good that is indispensable and essential to quality of life. It has always been marketed as easily accessible in terms of usage and availability although this may not genuinely be the case. In addition all consumption of electricity is indirect and consumption happens through the usage of other goods such as electrical appliances, entertainment etc. Distinction between some consumption that is foregone - lightning, listening to music, the radio etc. while other is postponed (dryer, washing machine etc.) 

Many things have been considered. Should we name different demand response mechanisms? 

This does not allow price spikes to form which may not be a good thing becuase they highlight what state the market is in. 

Something about tariffs. They are large may overshadow any price fluctuations and thus make it more difficult for residential consumers to adjust their consumption. 

\citep{kirschen2003demand}
 



