\label{sec:theory}
% Theoretical arguments in the literature closely related to your study
\subsection{Price formation in the spot market for electricity}
\label{subsec:t_price}


\subsection{Simultaneity of demand and supply}
\label{subsec:t_simultaneity}
Our data set contains all firms that have their hourly consumption charged by the corresponding spot market price, however, for big companies options exist to negotiate a less volatile settling or to buy financial products in order to increase price security. Our data does not cover firms that buy their electricity directly on the spot market or bilaterally by the electricity supplier on the so-called over-the-counter (OTC) market. It is a clear advantage that we know all of the demand in our firm data is directly subject to the spot price \citep{lijesen2007real}.


\subsection{Theory of demand-side response to electricity prices}
\label{subsec:t_demand}
Two obvious limitations to the flexibility of demand is that demand responses require knowledge of prices and that one can assume taking action in order to reduce demand at a certain hour carries a fixed cost of e.g. \$5 which makes is optimal for small consumers to ignore spot market except at extreme prices or with special tariffs \citep{wolak2011residential}. 





