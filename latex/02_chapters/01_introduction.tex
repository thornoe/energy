\label{sec:intro}
% Motivation of the study: why you focus on this particular issue
% Hypothesis and objective(s)
% Description of the background (theoretical and empirical) that lead you to propose the hypothesis
% Approach and summary of results: what is your strategy to check the hypothesis and the main result
% Structure of the paper
The focus of this paper is to estimate how the hour-by-hour electricity consumption responds to hourly electricity prices for wholesale and retail consumers. Estimating how price elasticities of demand of electricity has been an economic area of interest for a long time and increasingly so due to increased reliance on renewables in energy production. \bigskip \par
The electricity market has changed vastly over the past few decades in the direction of more competition and a larger share of intermittent, renewable energy production capacity. The climate crisis and the related ongoing political debate suggests that this will be equally, if not more, important in the near future. To policy makers and voters alike decarbonization is strongly linked to greater electrification, but this will only be true if this zero-emission, renewable energy production is able to meet demand.  Efficient and environmentally sustainable electricity provision implies that electricity production and thus electricity supply fluctuates according to weather conditions, namely wind speed and sunshine. Heterogeneity and changes over time in demand responses can help predicting potential demand flexibility in the future as this is the main limit for further increasing the reliance on wind and solar power along with the infeasibility of electricity storage. From a policy point of view this can also reveal the potential for time-of-use tariffs (and other demand responses) which are being regarded as the most cost-efficient tool for promoting a more sustainable electricity consumption cf. \citet{albadi2008summary}. \bigskip \par
Using hourly observations for 2016-2018 we contribute to the existing literature by analyzing 48 local grid companies in Denmark for which aggregate electricity consumption is split into wholesale (large and medium-sized firms) and retail consumption (small firms and households). Furthermore the problems of endogeneity resulting from the simultaneity of demand and supply mechanics is successfully handled by instrumenting the hourly spot price by the prognosis for wind power production as the current share of wind power greatly affects the marginal price of electricity in Denmark. To account for heterogeneity across the grid companies we estimate the price-elasticity both grid-by-grid using pooled 2SLS (P2SLS) and jointly while controlling for grid-level unobserved effects using random effects instrumental variables (REIV) estimation.
\par 
We obtain estimates ranging between -0.019 and -0.048 for wholesale consumers, while they range between 0 and -0.035 for retail consumers.  While consumption overall is inelastic it holds that wholesale consumption is more price-responsive than retail consumption which is in line with our theoretical predictions. Responsiveness is thus likely related to the degree of exposure to the real-time electricity price fluctuations. Results suggest that the prospects for using demand response (DR) mechanisms is limited and more centralized solutions may be called for when transitioning to an electricity market characterized by a large share of intermittent energy production. \todo{Tilføj noget med fremtidig forskning} \bigskip \par
The paper proceeds by giving a brief account of related studies in section \ref{sec:background}. Section \ref{sec:theory} covers the price formation in the electricity market by going into detail with the market itself in \ref{subsec:t_market}, the production side in \ref{subsec:t_production} and perspectives of the demand side in \ref{subsec:t_demand}. The data used for our empirical analysis is described in section \ref{sec:data} while we go into details with the econometric estimation method in section \ref{sec:empirical}. Results from the analysis are presented and discussed in \ref{sec:results}. Section \ref{sec:conclusion} concludes.
